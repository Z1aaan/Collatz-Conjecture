
    \documentclass{article}
    \usepackage[utf8]{inputenc}
    \usepackage{amsmath}
    \begin{document}
 %Title 
 \section{Collatz Sequence for \(n=15\) }
    %Collatz Function
    \[
        f(n)=
        \begin{cases}
        \frac{n}{2}, & n \mod 2=0
        \\
        3n+1, &n \mod 2=1
        \end{cases} \\
    \]
    % Path for given N
    \(\textbf{Path for f(15)}\\[3mm]f(n), n=15
 \\ 
 \Rightarrow 3(15) + 1
 \\ 
 \Rightarrow n=46
 \\[3mm] 
f(n), n=46
 \\ 
 \Rightarrow \frac{46}{2} 
 \\ 
 \Rightarrow n=23
 \\[3mm] 
f(n), n=23
 \\ 
 \Rightarrow 3(23) + 1
 \\ 
 \Rightarrow n=70
 \\[3mm] 
f(n), n=70
 \\ 
 \Rightarrow \frac{70}{2} 
 \\ 
 \Rightarrow n=35
 \\[3mm] 
f(n), n=35
 \\ 
 \Rightarrow 3(35) + 1
 \\ 
 \Rightarrow n=106
 \\[3mm] 
f(n), n=106
 \\ 
 \Rightarrow \frac{106}{2} 
 \\ 
 \Rightarrow n=53
 \\[3mm] 
f(n), n=53
 \\ 
 \Rightarrow 3(53) + 1
 \\ 
 \Rightarrow n=160
 \\[3mm] 
f(n), n=160
 \\ 
 \Rightarrow \frac{160}{2} 
 \\ 
 \Rightarrow n=80
 \\[3mm] 
f(n), n=80
 \\ 
 \Rightarrow \frac{80}{2} 
 \\ 
 \Rightarrow n=40
 \\[3mm] 
f(n), n=40
 \\ 
 \Rightarrow \frac{40}{2} 
 \\ 
 \Rightarrow n=20
 \\[3mm] 
f(n), n=20
 \\ 
 \Rightarrow \frac{20}{2} 
 \\ 
 \Rightarrow n=10
 \\[3mm] 
f(n), n=10
 \\ 
 \Rightarrow \frac{10}{2} 
 \\ 
 \Rightarrow n=5
 \\[3mm] 
f(n), n=5
 \\ 
 \Rightarrow 3(5) + 1
 \\ 
 \Rightarrow n=16
 \\[3mm] 
f(n), n=16
 \\ 
 \Rightarrow \frac{16}{2} 
 \\ 
 \Rightarrow n=8
 \\[3mm] 
f(n), n=8
 \\ 
 \Rightarrow \frac{8}{2} 
 \\ 
 \Rightarrow n=4
 \\[3mm] 
f(n), n=4
 \\ 
 \Rightarrow \frac{4}{2} 
 \\ 
 \Rightarrow n=2
 \\[3mm] 
f(n), n=2
 \\ 
 \Rightarrow \frac{2}{2} 
 \\ 
 \Rightarrow n=1
 \\[3mm] 

    \section{Credits and References Shit}
    %Maybe add link to gitrepo and other shit, idk
    This is created using Collatzer (https://github.com/Z1aaan/Collatzer).
    Created By: Z1aaan
    
    README:
    A program created to visualize and simulate a user-given value for \textit{N} 
    and see what happens when it is put under the Collatz function.
    \end{document}